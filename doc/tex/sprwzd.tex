\documentclass[12pt,a4paper,notitlepage]{article}

\usepackage{latexsym}
\usepackage[MeX]{polski}
\usepackage[utf8]{inputenc}
\usepackage{graphicx}
\usepackage{listings}
\usepackage{amsmath}

\usepackage[top=2cm, bottom=2cm, left=3cm, right=3cm]{geometry}

\makeatletter

\newcommand{\linia}{\rule{\linewidth}{0.4mm}}

\renewcommand{\maketitle}{\begin{titlepage}

    \vspace*{1cm}

    \begin{center}\small

   	 Politechnika Warszawska\\
    	Wydział Mechatroniki\\
 	Instytut Automatyki i Robotyki
    \end{center}
    \vspace{3cm}


    \begin{center}

	\small
	Programowanie w języku Java \\
	Projekt ćwiczeniowy \\
	
      \LARGE \textsc{Odtwarzacz i edytor muzyki}

         \end{center}


    \vspace{5.5cm}

    \begin{flushright}

    \begin{minipage}{5cm}

    \textit{\small Autorzy:}\\

    \normalsize \textsc{Konrad Traczyk} \par
    \normalsize \textsc{Jakub Szlendak} \par

    \end{minipage}

    \vspace{5cm}

     {\small Prowadzący:}\\
 
         {prof. dr hab. Barbara Siemiątkowska}\\	
         {mgr. inż Irina Gorbenko}

     \end{flushright}

    \vspace*{\stretch{6}}

    \begin{center}

    \@date

    \end{center}

  \end{titlepage}%

}

\makeatother

\begin{document}

\maketitle

\section{Wstęp}

Przedmiotem projektu było zaprojektowanie aplikacji odtwarzającej pliki muzyczne z funkcją edycji w języku programowania Java. Ustalono następujące wymagania co do funkcjonalności:
\begin{itemize}
 \item odtwarzanie plików w formacie MP3 i WAVE
 \item edycja plików w obydwu formatach
 \item realizacja podstawowych operacji edycyjnych - skracanie ścieżki muzycznej, wycinanie fragmentu ścieżki, modyfikacja poszczególnych próbek w ścieżce
 \item obsługa listy odtwarzania
 \item obsługa kilku trybów odtwarzania (pojedyncza piosenka, powtarzanie piosenki itp.)
\end{itemize}

\section{Implementacja aplikacji}



\end{document}


























 
